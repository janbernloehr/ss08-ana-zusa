\section{Differentialgleichungen}

\begin{defn}
Sei $I$ ein Intervall, $D\subset\R$ offen und $f:
I\times D\to \R$ stetig, dann heißt
\begin{align*}
\dot{x} = f(t,x),
\end{align*}
eine \emph{Differentialgleichung erster Ordnung} auf $I\times D$. Eine Lösung
dieser Differentialgleichung ist eine differenzierbare Abbildung
\begin{align*}
\ph(t) : J \to \R,\quad \varnothing \neq J \subset I,
\end{align*}
mit der Eigenschaft
\begin{align*}
\dot{\ph}(t) = f(t,\ph(t)),\quad t\in J.
\end{align*}
\end{defn}
\begin{defn}
Eine Differentialgleichung heißt \emph{autonom}, falls $f$ nicht explizit von
$t$ abhängt, also gilt $f: D\to\R$ mit $\dot{x} = f(x)$. Andernfalls
\emph{nichtautonom}.
\end{defn}
\begin{defn}
Ein zu einer Differentialgleichung gehörendes System
\begin{align*}
\dot{x} = f(t,x),\quad x(t_0) = x_0,
\end{align*}
nennt man \emph{Anfangswertproblem}, wobei $(t_0,x_0)\in I\times D$. Eine lokale
Lösung ist eine Lösung $\ph: I_0\to \R$ der Differentialgleichung mit
\begin{align*}
\ph(t_0) = x_0,\quad t_0\in I_0\subset I.
\end{align*} 
\end{defn}

\subsection{Lineare Differentialgleichungen - Der homogene Fall}
\begin{defn}
Eine \emph{lineare Differentialgleichung erster Ordnung} hat die Form
\begin{align*}
\dot{x} = a(t)x + b(t),
\end{align*}
mit auf $I$ stetigen Funktionen $a$ und $b$. Sie heißt \emph{homogen}, falls
$b=0$, andernfalls \emph{inhomogen}.
\end{defn}

\begin{prop}
Sei $a$ stetig auf $I$. Dann ist die allgemeine Lösung der
Differentialgleichung
\begin{align*}
\dot{x} = a(t)x,
\end{align*}
gegeben durch
\begin{align*}
\ph(t) = e^{A(t)}c, c \in\R
\end{align*}
mit einer beliebigen Stammfunktion $A$ von $a$. Sie existiert auf ganz $I$.
\end{prop}
% \begin{prop}[Quintesenz]
% Den homogen Fall löst man also, indem man eine Stammfunktion $A$ von $a$
% findet. Die allgemeine Lösung ist dann $\ph(t) = e^{A(t)}c$.
% \end{prop}
\begin{prop}
Die Gesamtheit aller Lösungen bildet einen
eindimensionalen Vektorraum
\begin{align*}
L_0 = \setdef{e^{A(t)}c}{c\in\R}.
\end{align*}
\end{prop}
\begin{prop}
Das Anfangswertproblem 
\begin{align*}
\dot{x} = a(t)x, \quad x(t_0) = x_0,
\end{align*}
besitzt auf $I$ die eindeutige Lösung
\begin{align*}
\ph(t) = \exp\left(\int_{t_0}^t a(s)\ds\right)x_0.
\end{align*}
\end{prop}

\subsection{Lineare Differentialgleichungen - Der inhomogene Fall}
\begin{prop}
Sei $\ph_0$ eine partikuläre Lösung der inhomogenen Gleichung
\begin{align*}
\dot{x} = a(t)x + b(t),
\end{align*}
dann hat jede andere Lösung die Form $\ph_0 + \ph$, mit einer Lösung $\ph$ der
homogenen Gleichung.
\end{prop}
\begin{prop}
Die Gesamtheit aller Lösungen bildet einen eindimensionalen affinen Vektorraum
\begin{align*}
L = \ph_0 + L_0 = \setdef{\ph_0 + e^{A(t)}c}{c\in\R}.
\end{align*}
\end{prop}
\begin{prop}
Sind $a$ und $b$ stetig auf $I$, dann ist die allgemeine Lösung von
\begin{align*}
\dot{x} = a(t)x + b(t),
\end{align*}
auf ganz $I$ erklärt und gegeben durch
\begin{align*}
\ph(t) = e^{A(t)}(c+c_0(t)),
\end{align*}
mit einer Stammfunktion $A$ von $a$ und $c_0$ von $e^{-A}b$.
\end{prop}
\begin{prop}
Sind $a$ und $b$ stetig auf $I$, dann besitzt das Anfangswertproblem
\begin{align*}
\dot{x} = a(t)x + b(t),\quad \ph(t_0) = x_0,
\end{align*}
auf $I$ die eindeutige Lösung
\begin{align*}
\ph(t) = e^{A(t)}\left(c + \int_{t_0}^t e^{-A(s)}b(s)\ds \right), \quad
 A(t) = \int_{t_0}^t a(s)\ds.
\end{align*}
\end{prop}

\begin{prop}[Prozedur]
Gegeben sei das Anfangswertproblem
\begin{align*}
\dot{x} = a(t)x + b(t), \quad x(t_0) = x_0.
\end{align*}
Zunächst löst man die homogene Gleichung $\dot{x} = a(t)x$, und erhält damit
\begin{align*}
\ph(t) = e^{A(t)}c,\quad A(t) = \int a(s)\ds.
\end{align*}
Mit Hilfe von $A(t)$ ergibt sich
\begin{align*}
c_0 = \int e^{-A(s)}b(s)\ds,
\end{align*}
und somit kann die inhomogene Gleichung gelöst werden
\begin{align*}
\ph(t) = e^{A(t)}(c + c_0).
\end{align*}
Nun setzt man $\ph(t_0) = x_0$ und löst nach $c$ auf.
\end{prop}

\subsection{Separierbare Differentialgleichungen}

\begin{defn}
Eine Differentialgleichung der Form
\begin{align*}
\dot{x} = g(t)h(x),
\end{align*}
mit stetigen Funktionen $g: I\to\R, h: J\to \R$ heißt \emph{separierbar}.
\end{defn}

\begin{prop}
Sind $g,h$ stetig und $x_0$ Nullstelle von $h$, so ist die konstante Funktion
$\ph\equiv x_0$ eine Lösung des Anfangswertproblems
\begin{align*}
\dot{x} = g(t)h(x),\quad \ph(t_0) = x_0.
\end{align*}
Ist $h$ lipschitz, so ist $\ph$ auch die einzige Lösung.
\end{prop}
\begin{prop}
Seien $g,h$ stetig auf $I$ bzw. $J$ und $h(x)\neq 0$ für $x\in J$. Dann
existiert genau eine lokale Lösung $\ph: I_0\to\R$ des Anfangswertproblems
\begin{align*}
\dot{x} = g(t)h(x),\quad \ph(t_0) = x_0,
\end{align*}
mit $t_0\in I$ und $x_0\in J$. Diese erfüllt die Gleichung
\begin{align*}
H(\ph(t)) = G(t),\quad t\in I_0,
\end{align*}
wobei
\begin{align*}
H(x) := \int_{x_0}^x \frac{\ds}{h(s)},\quad G(t) = \int_{t_0}^t
g(s)\ds.
\end{align*}
\end{prop}
\begin{prop}[Prozedur]
Gegeben sei das Anfangswertproblem
\begin{align*}
\dot{x} = g(t)h(x),\quad \ph(t_0) = x_0.
\end{align*}
Zunächst prüft man $h(x)$ auf Nullstellen $n_i$. Gilt $n_i = x_0$ für ein $i$
und ist $h(x)$ lipschitz, dann ist die einzige Lösung $\ph(t) \equiv x_0$.

Andernfalls separiert man die Variablen $t$ und $x$ und erhält
\begin{align*}
& \dot{x} = \frac{\dx}{\dt} = g(t)h(x)\\
\Leftrightarrow\;&\frac{\dx}{h(x)} = g(t)\dt\\
\Leftrightarrow\;&H(x) = \int \frac{\dx}{h(x)} = \int g(t)\dt = G(t).
\end{align*}
Kann man $H$ invertieren, erhält man als lokale Lösung
\begin{align*}
\ph(t) = x = H^{-1}(G(t)).
\end{align*}
Zuletzt setzt man $\ph(t_0) = x_0$ und löst nach $c$ auf. 
\end{prop}

\subsection{Homogene Differentialgleichung}

\begin{defn}
Eine Differentialgleichung der Form
\begin{align*}
\dot{x} = f(t,x),
\end{align*}
heißt \emph{homogen}, falls $f(\lambda t, \lambda x) = f(t,x)$. In diesem Fall
kann man $f(t,x)$ schreiben als $h\left(\frac{x}{t}\right)$.
\end{defn}

\begin{prop}
Sei $h$ stetig auf $I$ und $\frac{x_0}{t_0}\in I$. Eine Funktion $\ph : I_0\to
\R$ ist eine lokale Lösung des Anfangswertproblems
\begin{align*}
\dot{x} = h\left(\frac{x}{t}\right),\quad x(t_0) = x_0, 
\end{align*}
genau dann, wenn die Funktion
\begin{align*}
\psi: I_0\to\R,\quad \psi(t) = \frac{\ph(t)}{t}, 
\end{align*}
eine lokale Lösung des Anfangswertproblems
\begin{align*}
\dot{z} = \frac{h(z) - z}{t},\quad z(t_0) = \frac{x_0}{t_0},
\end{align*}
darstellt.
\end{prop}

\begin{prop}[Prozedur]
Gegeben sei ein Anfangswertproblem
\begin{align*}
\dot{x} = f(t,x),\quad x(t_0) = x_0.
\end{align*}
Um die Prozedur anwenden zu können, ist es notwendig zu überprüfen, ob $f(\lambda t,\lambda x)$ für $\lambda >
0$. In diesem Fall lässt sich $f$ schreiben als $f(t,x) = h\left(\frac{x}{t}\right)$.

Zunächst substituiert man $z = \frac{x}{t}$ und damit gilt
\begin{align*}
t\dot{z} - z = \dot{x} = h(z) \Leftrightarrow \dot{z} = \frac{h(z)+z}{t}.
\end{align*} 
Diese Differentialgleichung ist separierbar und eine allgemeine Lösung kann mit
der bekannten Prozedur bestimmt werden.

Anschließend muss die Lösung für $z$ mit $\ph(t) = x = zt$ rücksubstituiert, und
das Anfangswertproblem $\ph(t_0) = x_0$ gelöst werden.
\end{prop}