\section{Reihen}

\begin{defn}
Eine \emph{Reihe} ist ein Ausdruck der Form $\sum \limits_{k=1}^\infty
a_k$.
\end{defn}
\begin{defn}
Die endlichen Summen $\sum \limits_{k=1}^n a_k$ heißen \emph{$n$-te
Partialsummen} der Reihe.
\end{defn}

\begin{defn}
Eine Reihe heißt \emph{konvergent}, falls die Folge ihrer Partialsummen
konvergiert, andernfalls \emph{divergent}.
\end{defn}

\begin{defn}
Eine Reihe heißt \emph{absolut konvergent}, falls ihre Absolutreihe $\sum_k
\abs{a_k}$ konvergiert. Ist eine Reihe konvergent aber nicht absolut konvergent, heißt sie
\emph{bedingt konvergent}.
\end{defn}

\begin{defn}
Eine reelle Reihe $\sum_k b_k$ mit nichtnegativen Gleidern heißt
\emph{Majorante} der Reihe $\sum_k a_k$, wenn für fast alle Folgenglieder gilt $\abs{a_k} \le
b_k$.
\end{defn}

\subsection{Beispiele}
\begin{enumerate}
  \item Die geometrische Reihe
  $\sum \limits_{n=0}^\infty q^n$ konvergiert, falls $\abs{q} < 1$.
  \item Die harmonische Reihe
  $\sum \limits_{n=1}^\infty \frac{1}{n}$ divergiert.
  \item Die alternierende harmonische Reihe
  $\sum \limits_{n=1}^\infty \frac{(-1)^n}{n}$ ist bedingt konvergent.
  \item Die Zetafunktion
  $\sum \limits_{n=1}^\infty \frac{1}{n^r}$ konvergiert für $r>1$ und
  divergiert für $r\le1$.
  \item Die Exponentialreihe
  $\sum \limits_{n=0}^\infty \frac{z^n}{n!}$ ist für jedes $z\in\C$
  konvergent.
\end{enumerate}

\subsection{Sätze für Reihen in Banachräumen}

\begin{prop}[Nullfolgenkriterium]
Ist eine Reihe $\sum_k a_k$ konvergent, bilden ihre Glieder $a_k$ eine
Nullfolge.
\end{prop}

\begin{prop}[Cauchykriterium]
Ist eine Reihe $\sum_k a_k$ konvergent, dann gilt

$\forall \ep > 0 \exists N_0 \ge 0 \forall m,n \ge N : \norm{\sum
\limits_{k=n}^m a_k} < \ep$.
\end{prop}

\begin{prop}
Eine Reihe ist absolut konvergent, genau dann wenn ihre Absolutreihe beschränkt
ist. Jede absolut konvergente Reihe ist auch konvergent.
\end{prop}

\begin{prop}[Umordnungssatz]
Ist eine Reihe absolut konvergent, so ist auch jede Umordnung dieser Reihe
absolut konvergent.
\end{prop}

\begin{prop}[Majorantenkriterium]
Sei $\sum_k a_k$ eine Reihe. Existiert eine reelle konvergente Reihe $\sum_k
b_k$ so, dass für fast alle $k$ gilt $\norm{a_k} \le b_k$, dann ist $\sum_k a_k$
absolut konvergent.
\end{prop}

\begin{prop}[Wurzelkriterium]
Sei $\sum_k a_k$ eine Reihe. Existiert dann ein $q < 1$, sodass gilt
\begin{align*}
\sqrt[n]{\norm{a_n}} \le q, \text{ für fast alle } n,
\end{align*}
konvergiert die Reihe absolut. Gilt andernfalls
\begin{align*}
\sqrt[n]{\norm{a_n}} \ge 1,\text{ für unendlich viele } n,
\end{align*}
dann divergiert die Reihe.
\end{prop}

\begin{prop}[Quotientenkriterium]
Sei $\sum_k a_k$ eine Reihe. Existiert dann ein $q < 1$, sodass gilt
\begin{align*}
\frac{\norm{a_{n+1}}}{\norm{a_n}} \le q, \text{ für fast alle } n,
\end{align*}
konvergiert die Reihe absolut. Gilt andernfalls
\begin{align*}
\frac{\norm{a_{n+1}}}{\norm{a_n}} \ge 1, \text{ für unendlich viele } n,
\end{align*}
dann divergiert die Reihe.
\end{prop}

\subsection{Sätze für Reihen in $\R$}

\begin{prop}
Ist eine reelle Reihe konvergent aber nicht absolut konvergent, so existiert zu
jeder Zahl $s\in\R$ eine Umordnung, sodass die Reihe gegen $s$
konvergiert.
\end{prop}

\begin{prop}[Leibnizkriterium]
Die alternierende Reihe $\sum_k (-1)^k a_k$ konvergiert, falls $(a_k)$ eine
monoton fallende Nullfolge ist.
\end{prop}

\begin{prop}[Verdichtungssatz]
Ist $(a_n)$ eine reelle, monoton fallende Nullfolge, dann sind die Reihen
$\sum_k a_k$ und $\sum_k 2^ka_{2^k}$ entweder beide konvergent oder beide
divergent.
\end{prop}