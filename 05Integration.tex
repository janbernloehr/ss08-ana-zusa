\section{Integration}

\begin{defn}
Eine \emph{Zerlegung} $Z$ eines Intervalls $[a,b]$ ist ein Tupel reeller Zahlen
mit
\begin{align*}
a = t_1 < \ldots < t_n = b.
\end{align*}
\end{defn}
\begin{defn}
Eine Funktion $\ph: [a,b]\to\R$ heißt \emph{Treppenfunktion}, wenn eine
Zerlegung und reelle Zahlen $c_1, \ldots, c_k$ existieren, sodass
\begin{align*}
\ph(x)\big|_{t_k,t_{k+1}} = c_k.
\end{align*}
Der Raum der Treppenfunktionen auf $[a,b]$ wird mit $T_a^b$ bezeichnet.
\end{defn}
\begin{defn}
Das \emph{Integral der Treppenfunktion} ist
\begin{align*}
J_a^b(\ph) = \sum \limits_{k=1}^{n} \ph(t_k)(t_{k} - t_{k-1}).
\end{align*}
\end{defn}
\begin{defn}
Der Abschluss von $T_a^b$ bezüglich der Supremumsnorm heißt \emph{Raum der
Regelfunktionen} auf $[a,b]$ und wird mit $R_a^b$ bezeichnet.

Die stetige Fortsetzung von $J_a^b$ heißt \emph{Cauchyintegral} auf $[a,b]$ und
wird mit $\int_a^b$ bezeichnet.
\end{defn}
\begin{defn}
Es sei $a<b\le\infty$, und die Funktion $f: [a,b)\to \R$ sei über jedem
kompakten Intervall $[a,c]\subseteq [a,b)$ integrabel. Existiert der Limes
\begin{align*}
\int_a^b f(x)\dx = \lim_{c\to b}\int_a^c f(x)\dx,
\end{align*}
so heißt er das \emph{uneigentliche Integral} von $f$ über $[a,b)$ und man sagt,
das uneigentliche Integral \emph{konvergiert}. Andernfalls sagt man, es
\emph{divergiert}.

Analoges gilt für $f: (a,b]\to\R$ und $-\infty \le a < b$.
\end{defn}
\begin{defn}
Das Integral $\int_a^b f(x)\dx$ heißt \emph{absolut konvergent}, falls
das \emph{Absolutintegral}
\begin{align*}
\int_a^b \abs{f(x)}\dx
\end{align*}
existiert.
\end{defn}

\subsection{Sätze für Treppenfunktionen}
\begin{prop}
$T_a^b \le B_a^b$.
\end{prop}
\begin{prop}
Das Integral $J_a^b(\ph)$ hängt nicht von der Darstellung von $\ph$ ab.
\end{prop}
\begin{prop}
Das Funktional
\begin{align*}
J_a^b : T_a^b \to \R,\quad\ph \mapsto J_a^b(\ph),
\end{align*}
hat die Eigenschaften.
\begin{enumerate}
  \item Lineartiät. $J_a^b(\lambda \ph + \mu \psi) = \lambda J_a^b(\ph) + \mu
  J_a^b(\psi)$.
  \item Monotonie. $\ph \le \psi \Rightarrow J_a^b(\ph) \le J_a^b(\psi)$.
  \item Normiertheit. $\ph\big|_{(a,b)} = c \Rightarrow J_a^b(\ph) = c(b-a)$.
  \item Lipschitzstetigkeit mit $L$-Konstante $(b-a)$.
\end{enumerate}
\end{prop}

\subsection{Sätze für das Cauchyintegral}
\begin{prop}
Das Cauchyintegral hat die selben Eigenschaften wie das Integral der
Treppenfunktionen.
\end{prop}
\begin{prop}[Intervalladditivität]
Vereinbart man $\int_a^b f(x)\dx = -\int_b^a f(x)\dx$, dann gilt auf einem
Intervall, das $a,b,c$ umfasst
\begin{align*}
f\in R_a^b \Leftrightarrow f\in R_a^c \text{ und } f\in
R_c^b
\end{align*}
und außerdem
\begin{align*}
\int_a^b f(x)\dx = \int_a^c f(x)\dx + \int_c^b f(x)\dx.
\end{align*}
\end{prop}
\begin{prop}
Ist $f$ auf $[a,b]$ integrierbar, dann auch $\abs{f}$ und es gilt
\begin{align*}
\abs{\int_a^b f(x)\dx} \le \int_a^b \abs{f(x)}\dx.
\end{align*}
\end{prop}
\begin{prop}[Riemann'sches Lemma]
Ist $f$ auf $I$ integrierbar und in $c\in[a,b]$ stetig, dann gilt
\begin{align*}
\lim\limits_{h\to0} \frac{1}{h} \int_{c}^{c+h} f(x) \dx = f(c).
\end{align*}
\end{prop}
\begin{prop}[Mittelwertsatz der Integralrechnung]
Sei $f$ auf $[a,b]$ stetig, $p$ auf $[a,b]$ integrabel und $p\ge0$, dann gibt es
ein $c\in[a,b]$, sodass
\begin{align*}
\int_{a}^b (fp)(x) \dx = f(c)\int_a^b p(x)\dx.
\end{align*}
\end{prop}
% \begin{prop}
% Eine Regelfunktion $f\in R_a^b$ besitzt in jedem Punkt von $[a,b]$ links- und
% rechtsseitige Grenzwerte.
% \end{prop}
\begin{prop}
Eine Funktion $f: [a,b]\to\R$ ist eine Regelfunktion genau dann, wenn
sie in jedem Punkt links- und rechtsseitige Grenzwerte besitzt.
Insbesondere sind stetige und monotone Funktionen Regelfunktionen.
\end{prop}
\begin{prop}
Eine Regelfunktion besitzt nur abzählbar viele Unstetigkeitsstellen.
\end{prop}
\subsection{Sätze für das uneigentliche Integral}
\begin{prop}
Ist $a<b\le\infty$ und $f: [a,b)\to\R$, so sind folgende Aussagen äquivalent.
\begin{enumerate}
  \item Das uneigentliche Integral $\int\limits_a^b f(x)\dx$ konvergiert.
  \item Für jede Stammfunktion $F$ von $f$ existiert $\lim\limits_{x\to b}
  F(x)$.
  \item Es gilt das Cauchykriterium. Zu jedem $\ep > 0$ existiert ein
  $c\in[a,b)$, sodass
  \begin{align*}
  \abs{\int_u^v f(x) dx} < \ep, 
  \end{align*}
  für alle $u,v\in (c,b)$.
\end{enumerate}
\end{prop}
\begin{prop}[Satz von der absoluten Konvergenz]
Das uneigentliche Integral $\int_a^b f(x)\dx$ ist genau dann absolut
konvergent, wenn es beschränkt ist. In diesem Falle ist es auch
konvergent.
\end{prop}
\begin{prop}[Majorantenkriterium]
Gilt $\abs{f}\le g$ auf $[a,b)$ und existiert das Integral $\int_a^b g(x)\dx$,
so ist $\int_a^b f(x)\dx$ absolut konvergent.
\end{prop}
\begin{prop}
Nützliche Majoranten sind beispielsweise
\begin{align*}
  &\int_1^\infty \frac{\dx}{x^\alpha} < \infty \Leftrightarrow \alpha >
  1,\\
  &\int_0^1 \frac{\dx}{x^\alpha} < \infty \Leftrightarrow \alpha <
  1.
\end{align*}
\end{prop}