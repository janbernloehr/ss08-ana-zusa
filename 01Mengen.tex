\section{Mengen}

Im Folgenden sei $A$ immer eine Teilmenge eines normierten Raumes.

\begin{defn}
Eine Menge $A$ heißt beschränkt, falls
\begin{align*}
\sup\limits_{a\in A} \norm{a} < \infty.
\end{align*}
\end{defn}
\begin{defn}
Eine Menge $A$ heißt induktiv, falls
\begin{align*}
1\in A \text{ und } a \in A \Rightarrow a+1\in A.  
\end{align*}
\end{defn}
\begin{defn}
Eine Menge $A$ heißt \emph{offen}, wenn $A$ mit $a$ auch eine Umgebung von $a$
enthält.
\begin{align*}
\forall a\in A \exists \delta > 0 : B_\delta(a) \subset A.
\end{align*}
\end{defn}
\begin{defn}
Eine Menge $A$ heißt \emph{abgeschlossen}, wenn $A^c$
offen ist.
\end{defn}
\begin{defn}
Eine Menge $A$ heißt \emph{kompakt}, wenn jede Folge in $A$ eine in $A$
konvergente Teilfolge besitzt.
\end{defn}
\begin{defn}
Ein Punkt $a$ nicht notwendiger Weise $\in A$ heißt \emph{Häufungspunkt} von
$A$, wenn in jeder Umgebung von $a$ unendlich viele Punkte von $A$ liegen.
\end{defn}
\begin{defn}
Die Menge $A'$ aller Häufungspunkte von $A$ heißt \emph{abgeleitete
Menge}.
\end{defn}
\begin{defn}
Die Menge $A^{-} = A \cup A'$ heißt \emph{Abschluss} von $A$.
\end{defn}
\begin{defn}
Ein Punkt $a\in A$ heißt \emph{innerer Punkt} von $A$, wenn $A$ eine Umgebung
von $a$ enthält.
\end{defn}
\begin{defn}
Die Menge $A^\circ$ aller inneren Punkte von $A$ heißt \emph{offener Kern} von
$A$.
\end{defn}
\begin{defn}
Eine Menge $A$ heißt \emph{zusammenhängend}, falls es zu je zwei Punkten in $A$
eine ganz in $A$ liegende, differenzierbare Kurve gibt, die die zwei Punkte
verbindet.
\end{defn}
\begin{defn}
Eine Menge $\Omega$ heißt \emph{Gebiet}, wenn sie offen und zusammenhängend ist.
\end{defn}
\begin{defn}
Für zwei Punkte $u,v$ bezeichnet
\begin{align*}
[u,v] = \setdef{(1-t)u+tv}{t\in[0,1]}
\end{align*}
die \emph{Verbindungsstrecke}.
\end{defn}
\begin{defn}
Eine Menge $A$ heißt \emph{konvex}, falls für $u,v\in A$ auch $[u,v]\subset A$
ist.
\end{defn}
\begin{defn}
Seien $x_1, \ldots, x_n$ Punkte einer Menge, dann heißt die
Linearkombination
\begin{align*}
\sum\limits_{i=1}^n \lambda_i x_i,
\end{align*}
eine \emph{Konvexkombination}, falls $\lambda_i \ge 0$ und $\sum\limits_{i=1}^n
\lambda_i = 1$.
\end{defn}

\begin{defn}
Eine \emph{Norm} auf einem Vektorraum $V$ ist eine Abbildung
\begin{align*}
\norm{\cdot}: V\to \R,
\end{align*}
mit den Eigenschaften
\begin{enumerate}
  \item $\norm{x} \ge 0$ für alle $x\in V$ und $\norm{x} = 0$, wenn $x = 0$.
  \item $\norm{\lambda x} = \abs{\lambda}\norm{x}$ für alle $x\in
  V,\lambda\in\K$.
  \item $\norm{x + y} \le \norm{x} + \norm{y}$ für alle $x,y\in V$.
\end{enumerate}
\end{defn}
\begin{defn}
Ein \emph{Skalarprodukt} auf einem Vektorraum $V$ ist eine Abbildung
\begin{align*}
\lin{\cdot,\cdot}: V\times V\to \R,
\end{align*}
mit folgenden Eigenschaften für $x,y,z\in V$ und $\lambda,\mu\in\R$
\begin{enumerate}
  \item $\lin{x,x} \ge 0$,
  \item $\lin{x,y} = \lin{y,x}$,
  \item $\lin{\lambda x + \mu y,z} = \lambda\lin{x,z}+\mu\lin{y,z}$.
\end{enumerate}
\end{defn}

\subsection{Beispiele für Normen}

\begin{enumerate}
  \item Summennorm eines Vektors
  \begin{align*}
  \norm{x}_1 = \sum\limits_{i=1}^n\abs{x_i}.
  \end{align*}
  \item $p$-Norm eines Vektors
  \begin{align*}
  \norm{x}_p = \sqrt[p]{\sum\limits_{i=1}^n\abs{x_i}^p}.
  \end{align*}
  \item Supremumsnorm einer stetigen Abbildung $f: K\to W$
  \begin{align*}
  \norm{f}_\infty = \sup\limits_{x\in K} \abs{f(x)}.
  \end{align*}
  \item Operatornorm eines linearen Operators $A: V\to W$
  \begin{align*}
  \norm{A} = \sup\limits_{x\in V\setminus \{0\}} \frac{\abs{Ax}_W}{\abs{x}_V} =
  \sup\limits_{\abs{x}_V = 1} \abs{Ax}.
  \end{align*}
  \item $L_1$ Norm einer stetigen integrablen Abbildung $f$
  \begin{align*}
  \norm{f}_{L_1} = \int\limits_{-\infty}^\infty \abs{f(x)}\dx.
  \end{align*}
\end{enumerate}

\subsection{Sätze in einem normierten Raum}

\begin{prop}
Ist $N$ eine induktive Teilmenge von $\N$, so gilt $N = \N$.
\end{prop}
\begin{prop}
$\varnothing$ und $E$ sind offen und abgeschlossen.
\end{prop}
\begin{prop}
Die Vereinigung beliebig vieler offener Mengen ist offen.
\end{prop}
\begin{prop}
Der Durchschnitt endlich vieler offener Mengen ist offen.
\end{prop}
\begin{prop}
Die Vereinigung endlich vieler abgeschlossener Mengen ist
abgeschlossen.
\end{prop}
\begin{prop}
Der Durchschnitt beliebig vieler abgeschlossener Mengen ist
abgeschlossen.
\end{prop}
\begin{prop}
Die Vereinigung endlich vieler kompakter Mengen ist
kompakt.
\end{prop}
\begin{prop}
$A$ ist abgeschlossen genau dann, wenn $A$ alle seine
Häufungspunkte enthält, also genau dann, wenn $A=A^{-}$.
\end{prop}
\begin{prop}
$A$ ist offen genau dann, wenn alle Punkte von $A$ innere Punkte sind, also
genau dann, wenn $A=A^{\circ}$.
\end{prop}
\begin{prop}
Jede abgeschlossene Teilmenge einer kompakten Menge ist kompakt.
\end{prop}
\begin{prop}
Eine kompakte Menge ist abgeschlossen und
beschränkt.
\end{prop}
\begin{prop}
Eine Menge $K$ ist konvex genau dann, wenn sie mit den Punkten $x_1, \ldots, x_n
\in K$ auch jede Konvexkombination dieser Punkte enthält. 
\end{prop}
\begin{prop}
Auf endlichdimensionalen Vektorräumen sind alle Normen äquivalent.
\end{prop}
\begin{prop}[Bolzano Weierstraß]
Eine Teilmenge des $\R,\C,\R^n$ ist kompakt genau dann, wenn sie abgeschlossen
und beschränkt ist.
\end{prop}
\begin{prop}[Überdeckungslemma von Heine-Borell]
Sei $K$ kompakt und $(I_\lambda)_{\lambda\in\Lambda}$ eine beliebige Familie
offener Intervalle. Gilt 
\begin{align*}
\bigcup \limits_{\lambda\in\Lambda} I_\lambda \supset K,
\end{align*}
so existieren endlich viele Umgebungen $I_1, \ldots, I_m$, sodass
\begin{align*}
\bigcup \limits_{1\le i\le m} I_i \supset K.
\end{align*}
\end{prop}