\section{Matritzen}

\begin{defn}
Eine Matrix $A\in S(n)$ heißt
\begin{enumerate}
  \item \emph{positiv definit}, geschrieben $A > 0$, falls $\lin{Av,v} > 0,
  \forall v\neq 0$,
  \item \emph{positiv semidefinit}, geschrieben $A \ge 0$, falls $\lin{Av,v} \ge
  0, \forall v$,
  \item \emph{negativ definit} $A < 0$, falls $-A > 0$,
  \item \emph{negativ semidefinit} $A\le 0$, falls $-A \ge 0$,
  \item sonst \emph{indefinit} $A \lessgtr 0$.
\end{enumerate}
\end{defn}

\begin{prop}
Die Abbildung $\det: \R^{n\times n} \to \R, A\mapsto \det A$ ist stetig
differenzierbar auf $\R^{n\times n}$.
\end{prop}
\begin{prop}
Ist $A\in S(n)$, dann sind folgende Aussagen äquivalent
\begin{enumerate}
  \item $A$ ist positiv definit.
  \item Es gibt ein $\lambda > 0$, sodass $\lin{Av,v} \ge \lambda\abs{v}^2$.
  \item Es gibt ein $\lambda > 0$, sodass $A -\lambda E \ge 0$.
\end{enumerate}
\end{prop}
\begin{prop}
Sei $A\in S(n)$ und $\lambda_1\le\ldots\le\lambda_n$ seien Eigenwerte, dann gilt
\begin{enumerate}
  \item $A > 0 \Leftrightarrow \lambda_1 > 0$ und $A \ge 0 \Leftrightarrow
  \lambda_1 \ge 0$.
  \item $A < 0 \Leftrightarrow \lambda_n < 0$ und $A \le 0 \Leftrightarrow
  \lambda_n \le 0$.
  \item $A \lessgtr 0 \Leftrightarrow \lambda_1\lambda_n < 0$. 
\end{enumerate}
\end{prop}
\begin{prop}
Eine Matrix $A\in S(n)$ ist positiv definit genau dann, wenn alle ihre
Haupt-Unterdeterminanten positiv sind. Sie ist positiv semidefinit genau dann,
wenn diese nicht negativ sind.
\end{prop}
\begin{prop}
Sei $M:\Omega\to S(n)$ eine steige matrixwertige Abbildung und $M(a) > 0$. Dann
existiert eine Umgebung $U$ von $a$ derart, dass
\begin{align*}
M(x) > 0, \forall x\in U.
\end{align*}
\end{prop}