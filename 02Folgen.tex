\section{Folgen}

\begin{defn}
Eine \emph{Folge} in einer Menge $M$ ist eine Funktion $\N \to M$.
\end{defn}
\begin{defn}
Ist $(a_n)$ eine Folge in $M$ und $(n_k)$ eine strikt wachsende Folge
natürlicher Zahlen, dann heißt $(a_{n_k})$ \emph{Teilfolge} von $(a_n)$ und
$(n_k)$ \emph{Auswahlfolge}.
\end{defn}
\begin{defn}
Ist $(a_n)$ eine Folge und $W = \setdef{a_n}{n\in\N}$, dann heißt $(a_n)$
\emph{beschränkt}, falls $W$ beschränkt ist.
\end{defn}
\begin{defn}
Eine Folge $(a_n)$ in $(M,\norm{\cdot})$ heißt \emph{konvergent gegen $a$}
bezüglich $\norm{\cdot}$, wenn
\begin{align*}
\forall &\ep > 0 \exists N_0 >
0 \forall n \ge N_0:
\norm{a_n - a} < \ep,\\
& n \ge N_0 \Rightarrow \norm{a_n - a} < \ep.
\end{align*}
\end{defn}
\begin{defn}
Eine Folge $(a_n)$ heißt \emph{uneigentlich konvergent} gegen $\infty$ bzw.
$-\infty$, wenn in jeder Umgebung von $\infty$ bzw. $-\infty$ fast alle
Folgenglieder liegen.
\end{defn}
\begin{defn}
Eine Folge, die nicht konvergiert, heißt \emph{divergent}.
\end{defn}
\begin{defn}
Eine Folge $(a_n)$ heißt \emph{Cauchyfolge}, wenn
\begin{align*}
\forall &\ep > 0 \exists N_0 > 0 \forall n,m \ge N_0 :
\norm{a_n - a_m} < \ep,\\
&n,m \ge N_0 \Rightarrow \norm{a_n - a_m} < \ep.
\end{align*}
\end{defn}
\begin{defn}
Für eine Folge $(a_n)$ in $M\subset E$ heißt ein Element $a\in E$
\emph{Häufungspunkt}, wenn in jeder Umgebung von $a$ unendlich viele
Folgenglieder liegen.
\end{defn}
\begin{defn}
Den Raum der beschränkten Folgen nennt man $c$, den Raum der Nullfolgen $c_0$.
\end{defn}

\subsection{Allgemeine Sätze für Folgen in normierten Räumen}

\begin{prop}
Der Grenzwert einer konvergenten Folge ist eindeutig bestimmt.
\end{prop}
% \begin{prop}
% Eine konvergente Folge ist beschränkt.
% \end{prop}
\begin{prop}
Gilt $\norm{a_n - a} \le \norm{b_n}$ für eine Nullfolge $b_n$ und fast
alle $n$, dann ist $(a_n)$ konvergent mit Grenzwert $a$.
\end{prop}
\begin{prop}
Jede konvergente Folge ist eine Cauchyfolge.
\end{prop}
\begin{prop}
Jede Cauchyfolge ist beschränkt.
\end{prop}
\begin{prop}
Besitzt eine Cauchyfolge eine konvergente Teilfolge, dann ist die gesamte
Folge konvergent mit dem selben Grenzwert.
\end{prop}
\begin{prop}
Sind $(a_n)$ und $(b_n)$ Folgen mit $a_n \to a$, $b_n \to b$, so gilt für
$\lambda,\mu\in \K$
\begin{align*}
\lambda a_n + \mu b_n \to \lambda a + \mu b
\end{align*}
\end{prop}
\begin{prop}
Ist $(b_n)$ beschränkt und $(c_n)$ eine Nullfolge, so gilt $\lin{b_n,c_n} \to
0$.
\end{prop}
\begin{prop}
Sind die Folgen $(a_n)$ und $(b_n)$ konvergent mit Grenzwert $a$ und $b$, dann
gilt $\lin{a_n,b_n} \to \lin{a,b}$.
\end{prop}

\subsection{Sätze für Folgen in Banachräumen}

\begin{prop}
Eine Folge ist genau dann konvergent, wenn sie eine Cauchyfolge ist.
\end{prop}
\begin{prop}
Die Folgenräume $c$ und $c_0$ mit der Supremumgsnorm sind vollständig.
\end{prop}

\subsection{Sätze für Folgen im $\R^n$}

\begin{prop}
Eine Folge konvergiert genau dann, wenn sie komponentenweise konvergiert.
\end{prop}
\begin{prop}
Jede beschränkte Folge besitzt eine konvergente Teilfolge.
\end{prop}

\subsection{Sätze in $\R$}

\begin{prop}
Sind $(a_n),(b_n)$ konvergent und gilt $a_n \le b_n$ für unendlich viele $n$,
dann ist auch $\lim a_n \le \lim b_n$.
\end{prop}
\begin{prop}
Allgemeine Grenzwertsätze
\begin{enumerate}
  \item $a_n \to \infty \text{ und } b_n \ge c \Rightarrow a_n + b_n \to
  \infty$,
  \item $a_n \to \infty \text{ und } b_n \ge c > 0 \Rightarrow a_nb_n \to
  \infty$,
  \item $\abs{a_n} \to \infty  \Rightarrow a_n^{-1} \to 0$,
  \item $a_n \to 0 \text{ und } a_n > 0 \Rightarrow a_n^{-1} \to \infty$.
\end{enumerate}
\end{prop}
\begin{prop}
Für jede reelle Zahl $q$ mit $\abs{q} < 1$ gilt $q^n \to 0$.
\end{prop}
\begin{prop}
Für jede reelle Zahl $q$ mit $\abs{q} < 1$ und jedes $p\in\Z$ gilt $n^zq^n \to
0$.
\end{prop}
\begin{prop}
Für jede reelle Zahl $a$ gilt $\frac{a^n}{n!} \to
0$.
\end{prop}
\begin{prop}
$\sqrt[n]{n} \to 1$.
\end{prop}
\begin{prop}[Erweiterter Annäherungssatz]
In jeder nichtleeren Menge $A$ existiert eine Folge, die gegen $\sup A$
konvergiert. Entsprechendes gilt für $\inf A$.
\end{prop}
\begin{prop}[Erweiterter Satz von der monotonen Konvergenz]
Jede monotone Folge konvergiert gegen einen eigentlichen oder uneigentlichen
Grenzwert.
\end{prop}
\begin{prop}[Erweiterter Satz von Bolzano Weierstrass]
Jede reelle Folge besitzt eine eigentlich oder uneigentlich konvergente
Teilfolge.\\
Jede reelle Folge besitzt einen eigentlichen oder uneigentlichen
Häufungswert.
\end{prop}