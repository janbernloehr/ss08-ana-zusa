\section{Kurven und Wege}

Im Folgenden sei $I=[a,b]$ ein kompaktes Intervall von $\R$ und $E$ Bannachraum.

\begin{defn}
Eine \emph{Kurve} ist eine $C^0$-Abbildung $\gamma: I\to E$.

Ihr Bild $\gamma(I) = \setdef{\gamma(t) \in E}{t\in I}$
heißt \emph{Spur}.
\end{defn}
\begin{defn}
Eine Kurve $\gamma: [a,b]\to E$ heißt \emph{geschlossen}, falls $\gamma(a) =
\gamma(b)$.
\end{defn}
\begin{defn}
Eine Kurve $\gamma: [a,b]\to E$ heißt \emph{einfach} oder
\emph{doppelpunktfrei}, falls $\gamma\big|_{(a,b]}$ und $\gamma\big|_{[a,b)}$ injektiv sind.
\end{defn}
\begin{defn}
Eine Kurve $\gamma: I\to E$ heißt \emph{differenzierbar im Punkt} $t_0\in
I$, wenn es einen Vektor $v\in E$ und eine in $t_0$ stetige Abbildung $r: I \to
E$ mit $r(t_0) = 0$ gibt, sodass
\begin{align*}
\gamma(t) = \gamma(t_0) + v(t-t_0) + r(t)(t-t_0).
\end{align*}
In diesem Fall heißt $v$ die \emph{erste Ableitung} von $\gamma$ im Punkt $t_0$
und wird mit $\dot{\gamma}(t_0)$ bezeichnet.
\end{defn}
\begin{defn}
Ist $\gamma: I\to E$ in $t_0$ differenzierbar, dann bezeichnet man
$\dot{\gamma}(t_0)$ als \emph{Tangentialvektor} von $\gamma$ im Punkt
$\gamma(t_0)$ und $\abs{\dot{\gamma}(t_0)}$ als seine \emph{momentan
Geschwindigkeit}.
\end{defn}
\begin{defn}
Ist $\dot{\gamma}(t_0)\neq0$, so ist die \emph{Tangente} an $\gamma$ im Punkt
$t_0$ gegeben durch
\begin{align*}
\alpha(t) = \gamma(t_0) + \dot{\gamma}(t_0)(t-t_0).
\end{align*}
\end{defn}
\begin{defn}
Eine $C^1$-Kurve heißt \emph{regulär}, wenn ihre Ableitung nirgends
verschwindet.
\end{defn}
\begin{defn}
Die \emph{Länge} einer Kurve $\gamma\in C^0(I,E)$ ist
\begin{align*}
L_I(\gamma) = \sup\limits_T \sum_T \norm{\gamma(t_k)-\gamma(t_{k-1})}.
\end{align*} 
\end{defn}
\begin{defn}
$\gamma\in C^0(I,E)$ heißt \emph{rektifizierbar}, falls $L_I(\gamma) <
\infty$.
\end{defn}
\begin{defn}
Sei $\gamma\in C^0(I,E)$. Die \emph{Längenfunktion} ist definiert als
\begin{align*}
\lambda : I \to \R,\quad \lambda(t) = L_{[a,t]}(\gamma).
\end{align*}
\end{defn}
\begin{defn}
Eine \emph{Parametertransformation} ist eine bijektive stetige Abbildung $\ph:
I\to I^*$ eines Intervalls $I$ auf ein Intervall $I^*$. Ist $\ph$ monoton steigend
heißt $\ph$ \emph{orientierungstreu}, andernfalls \emph{orientierungsumkehrend}.
\end{defn}
\begin{defn}
Zwei Kurven $\gamma \in C^0(I,E)$ und $\gamma^*\in C^0(I^*,E)$ heißen
\emph{topologisch äquivalent} geschrieben $\gamma\sim\gamma^*$, falls eine
orientierungstreue Parametertransformation $\ph$ existiert, sodass
\begin{align*}
\gamma = \eta\circ\ph.
\end{align*}
\end{defn}
\begin{defn}
Ein \emph{stetiger Weg} $\omega$ in $E$ ist eine Klasse topologisch äquivalenter
Kurven
\begin{align*}
\omega = [\gamma] = \setdef{\eta\in C^0(I,E)}{\gamma\sim\eta}.
\end{align*}
Jedes Element $\eta\in\omega$ heißt \emph{Parametrisierung} des Weges.
\end{defn}
\begin{defn}
Ein Weg $\omega = [\gamma]$ heißt \emph{einfach}, falls $\gamma$ einfach,
\emph{geschlossen}, falls $\gamma$ geschlossen und \emph{Jordanweg}, falls
$\gamma$ Jordankurve ist. Der \emph{Anfangs-} und \emph{Endpunkt} von $\omega$
ist der Anfangs- und Endpunkt von $\gamma$ und die \emph{Spur} von $\omega$ ist
die Spur von $\gamma$.
\end{defn}
\begin{defn}
Ein Weg heißt \emph{rektifizierbar}, falls er eine rektifizierbare
Parametrisierung besitzt. Seine Länge ist in diesem Fall die einer beliebigen
Parametrisierung.
\end{defn}
\begin{defn}
Ein Weg heißt \emph{glatt}, falls er eine reguläre Parametrisierung besitzt.
\end{defn}

\subsection{Sätze für Kurven in Banachräumen}
\begin{prop}
Für eine in $t_0\in I$ differenzierbare Kurve $\gamma: I\to E$ ist äquivalent
\begin{enumerate}
  \item Es gibt eine in $t_0$ stetige Abbildung $\ph: I\to E$ mit $\ph(t_0) =
  v$, sodass
  \begin{align*}
  \gamma(t) = \gamma(t_0) + \ph(t)(t-t_0).
  \end{align*}
  \item Es gilt
  \begin{align*}
  \lim\limits_{h\to0} h^{-1}(\gamma(t_0+h)-\gamma(t_0)) = v.
  \end{align*}
  \item Es gilt
  \begin{align*}
  \lim\limits_{t\to t_0}
  \frac{\abs{\gamma(t)-\gamma(t_0)-v(t-t_0)}}{\abs{t-t_0}} = 0.
  \end{align*}
\end{enumerate}
\end{prop}
\begin{prop}
Sei $\gamma\in C^0(I,E)$ und $t_0\in I$, dann wird durch
\begin{align*}
\Phi(t) = \int_{t_0}^t \gamma(t)\dt,
\end{align*}
eine stetig differenzierbare Funktion definiert mit
$\dot{\Phi}=\gamma$.
\end{prop}
\begin{prop}[Haupsatz der Differential und Integralrechnung]
Sei $\gamma\in C^0(I,E)$ und $\Gamma$ eine beliebige Stammfunktion auf
$[a,b]\subset I$, dann gilt
\begin{align*}
\Gamma(b) - \Gamma(a) = \int_a^b\gamma(t)\dt.
\end{align*}
Außerdem gilt für jede Kurve $\gamma\in C^1(I,E)$, dass $\gamma(b)-\gamma(a) =
\int_a^b\dot{\gamma}(t)\dt$ ist.
\end{prop}
\begin{prop}
Jede Kurve $\gamma\in C^1(I,E)$ ist lipschitz mit $L$-Konstante $M =
\max\limits_{t\in I} \norm{\dot{\gamma}(t)}_E$.
\end{prop}
\begin{prop}
Ist $\gamma\in C^0(I,E)$ lipschitz mit $L$-Konstante $M$, dann ist
$\gamma$ rektifizierbar und es gilt
\begin{align*}
L_I(\gamma) \le M\abs{I}.
\end{align*}
Insbesondere ist jede $C^1$ Kurve rektifizierbar.
\end{prop}
\begin{prop}
Die Längenfunktion ist additiv $L_{[a,c]}+L_{[c,b]} = L_{[a,b]}$.
\end{prop}
\begin{prop}
Ist $\gamma\in C^1(I,E)$, so ist die Längenfunktion stetig differenzierbar und
es gilt $\lambda'(t) = \norm{\dot{\gamma}(t)}_E$, sowie
\begin{align*}
\lambda(t) = \int_a^t \norm{\dot{\gamma}(t)}_E\dt.
\end{align*}
\end{prop}
\begin{prop}
Die euklidische Länge einer Kurve ist
\begin{align*}
L_I(\gamma) = \int_I
\sqrt{\dot{\gamma}_1^2(t)+\ldots+\dot{\gamma}_n^2(t)}\dt.
\end{align*}
\end{prop}
\begin{prop}
Seien $\gamma$ und $\gamma^*$ topologisch äquivalent. Ist $\gamma$
rektifizierbar, so auch $\gamma^*$ und die Längen sind gleich.
\end{prop}
\begin{prop}
Jeder glatte Weg besitzt eine Parametrisierung nach der Bogenlänge 
\begin{align*}
\eta: [0,l]\to E
\end{align*}
derart, dass $\norm{\dot{\eta}(t)}_E = 1$ für alle $t\in[0,l]$ mit $l =
L(\omega)$.
\end{prop}
\begin{prop}
Jede stückweise $C^1$ Kurve ist rektifizierbar und es gilt
\begin{align*}
L_I(\gamma) = \int_I \norm{\dot{\gamma}(t)}\dt.
\end{align*}
\end{prop}

\subsection{Sätze für Kurven im $\R^n$}

\begin{prop}[Jordanscher Kurvensatz]
Ist $\gamma: I\to\R^2$ geschlossenen und doppelpunktfrei, so besteht das
Komplement von $\Gamma = \gamma(I)$ genau aus zwei disjunkten,
zusammenhängenden Mengen $\Omega_i$ und $\Omega_o$, genannt das Innere und
das Äußere. Außerdem ist $\Omega_i$ beschränkt und $\Gamma$ bildet den Rand von
$\Omega_i$ und $\Omega_o$.
\end{prop}
\begin{prop}[Satz von Peano]
Es gibt stetige Abbildungen $\gamma: [0,1]\to[0,1]^2$, die surjektiv
sind.
\end{prop}
\begin{prop}
Eine Kurve $\gamma: I \to \R^n$ ist differenzierbar genau dann, wenn jede
Komponentenfunktion differenzierbar ist und es gilt
\begin{align*}
\dot{\gamma}(t) = (\dot{\gamma_1}(t),\ldots,\dot{\gamma_n}(t)).
\end{align*}
\end{prop}
\begin{prop}
Eine Kurve ist von der Klasse $C^r$, wenn jede Komponentenfunktion $C^r$
ist.
\end{prop}
\begin{prop}
Ist $\gamma: I\to\R^2$ eine reguläre $C^r$ Kurve, so ist die Spur von $\gamma$
lokal um jeden Punkt der Graph einer $C^r$ Funktion.
\end{prop}
