\section{Mehrdimensionale Differentiation}

\begin{defn}
Eine Abbildung $f:V\hookrightarrow W$ heißt \emph{differenzierbar} in $a$, wenn
es eine lineare Abbildung $L$ und eine in $a$ stetige Abbildung $r: \Omega \to \R^m$ 
gibt, sodass gilt
\begin{align*}
f(a+h) = f(a) + Lh + r(a+h)\abs{h}.
\end{align*}
\end{defn}
\begin{defn}
Sei $f:V\hookrightarrow W$, $a$ aus dem Definitionsbereich und $v\in V$
beliebig. Existiert die Ableitung
\begin{align*}
D_vf(a) = \frac{d}{dt} f(a+tv)\big|_{t=0},
\end{align*}
so heißt $D_vf(a)$ die \emph{Richtungsableitung} von $f$ an der Stelle $a$ in
Richtung $v$.
\end{defn}
\begin{defn}
Sei $f:\R^n\hookrightarrow W$, $a$ aus dem Definitionsbereich $1\le i\le n$.
Existiert die Ableitung
\begin{align*}
D_if(a) = \frac{d}{dt} f(a+te_i)\big|_{t=0} = \frac{\partial}{\partial_{x_i}}
f(a) = \partial_{x_i}f(a) = f_{x_i}(a),
\end{align*}
so nennt man $D_if(a)$ die \emph{partielle Ableitung} von $f$ nach $i$ im Punkt
$a$.
\end{defn}
\begin{defn}
Die \emph{Operatornorm} einer linearen Abbildung $L: V\to W$ ist definiert als
\begin{align*}
\norm{L} = \sup\limits_{v\neq0} \frac{\abs{Lv}_W}{\abs{v}_V} =
\sup\limits_{\abs{v}=1} \abs{Lv}_W.
\end{align*}
\end{defn}
\begin{defn}
Sei $V$ ein Hilbertraum und $f: V\hookrightarrow W$ in $x$ differenzierbar.
Dann ist der \emph{Gradient} der eindeutig bestimmte und mit $\nabla f(x)$
bezeichnete Vektor, für den gilt $Df(x)h = \lin{\nabla f(x),h}$.
\end{defn}
\begin{defn}
Ist $f: \R^n\hookrightarrow\R$ stetig differenzierbar, so heißt der Graph der
Abbildung
\begin{align*}
T: \R^n\hookrightarrow\R,\quad z(x) = f(a)+\lin{\nabla f(a),x-a},
\end{align*} die \emph{Tangentialebene} an den Graphen von $f$ im Punkt $a$.
\end{defn}
\begin{defn}
Die $r$-te partielle Ableitung einer Abbildung $f: \R^n\hookrightarrow\R^m$,
$D_{r}D_{r-1}\ldots D_1 f(x)$ ist rekursiv erklärt durch 
$D_{r}D_{r-1}\ldots D_1 f(x) = D_{r}(D_{r-1}\ldots D_1 f(x))$.
\end{defn}

\subsection{Beispiele}
\begin{enumerate}
  \item Eine affine Abbildung $f: V\hookrightarrow W, x \mapsto Ax+b$ ist
  überall differenzierbar und es gilt $Df(x) = A$.
  \item Die Abbildung $g: V\hookrightarrow W, x \mapsto
  \lin{Ax,x}+b$ ist überall differenzierbar und es gilt $Df(x) =
  \lin{Ax,\cdot} + \lin{A^tx,\cdot}$, ist $A$ symmetrisch gilt sogar $Df(x) =
  2Ax$.
  \item Für $f(x,y) = \frac{1}{2}(x^2-y^2)$ ist $\nabla f(x,y) = (x,-y)^T$.
  \item Für die Abbildung $f(x,y) = \frac{2xy}{x^2+y^2}$ existieren im 
  Nullpunkt beide partiellen Ableitungen und es gilt $f(x,0) = f(0,y) = 0$, sie
  ist jedoch nicht total differenzierbar. Sie ist in $0$ nicht einmal
  stetig, denn
  \begin{align*}
  f(t\cos \ph,t\sin\ph) = \frac{2\cos\ph\sin\ph}{\cos^2 \ph + \sin^2\ph} =
  2\sin\ph.
  \end{align*}
\end{enumerate}

\subsection{Sätze für Funktionen $f: V \hookrightarrow W$}

\begin{prop}
Für eine Abbildung $f:\Omega \to W$ sind folgende Aussagen äquivalent
\begin{enumerate}
  \item $f$ ist differenzierbar in $a$ mit $Df(a) = L$.
  \item $f(a+h) = f(a) + Lh + o(h)$. 
  \item $\lim \abs{h}^{-1}\abs{f(a+h)-f(a)-Lh} = 0$.
\end{enumerate}
\end{prop}
\begin{prop}
Ist $f$ in $a$ differenzierbar, so ist $f$ in $a$ stetig und $Df(a)$ ist
eindeutig bestimmt.
\end{prop}
\begin{prop}
Die Differentiation ist eine lineare Operation und $D$ ist ein linearer
Operator.
\end{prop}
\begin{prop}[Kettenregel]
Ist $f: U\hookrightarrow V$ in $a$ differenzierbar und $g:
V\hookrightarrow W$ in $f(a)$, so ist auch $g\circ f: U\hookrightarrow W$ in
$a$ differenzierbar und es gilt
\begin{align*}
D(g\circ f)(a) = Dg(f(a))Df(a).
\end{align*}
\end{prop}
\begin{prop}
Ist $\Omega$ wegzusammenhängend, dann ist eine Abbildung $f: \Omega\to W$
konstant genau dann, wenn $Df(x) \equiv 0$ ist.
\end{prop}
\begin{prop}
Ist $f: U\hookrightarrow W$ differenzierbar, so gilt
\begin{align*}
D_vf(a) = Df(a)v.
\end{align*}
\end{prop}
\begin{prop}
Ist $f: \R^n\hookrightarrow\R^m$ in $a$ differenzierbar, so existieren alle
Richtungsableitungen von $f$ und die totale Ableitung ist durch die Jacobimatrix darstellbar
\begin{align*}
Df(a) = \begin{pmatrix}
        \partial_{x_1}f_1(a) & \ldots & \partial_{x_n}f_1(a)\\
        \vdots & \ddots & \vdots\\
        \partial_{x_1}f_m(a) & \ldots & \partial_{x_n}f_m(a)
        \end{pmatrix}.
\end{align*}
\end{prop}
\begin{prop}
Ist $f:\R^n\hookrightarrow\R^m$ in $a$ differenzierbar, so ist
\begin{align*}
D_vf(a) = \sum\limits_{i=1}^n D_if(a)v_i = \sum\limits_{i=1}^n
f_{x_i}(a)v_i.
\end{align*}
\end{prop}
\begin{prop}
Existieren sämtliche partiellen Ableitungen von $f: \Omega\to\R^m$ und sind
diese auf $\Omega$ stetig, so ist $f$ auf $\Omega$ differenzierbar und die
Ableitung ist stetig.
\end{prop}
\begin{prop}[Lemma von Hadamard]
Ist $f \in C^1(\Omega,W)$ und $[u,v]\subset\Omega$, so gilt
\begin{align*}
f(v)-f(u) = L(v-u), \text{ mit } L = \int_{0}^1 Df((1-t)u
+tv)\dt.
\end{align*}
\end{prop}
\begin{prop}[Schrankensatz]
Ist $f \in C^1(\Omega,W)$ und $[u,v]\subset\Omega$, so gilt
\begin{align*}
\abs{f(v)-f(u)}_W \le \max\limits_{z\in[u,v]}\norm{Df(z)}\abs{v-u}_W.
\end{align*}
\end{prop}
\begin{prop}
Ist $f \in C^1(\Omega,W)$, so ist $f$ lokal lipschitz.
\end{prop}
\begin{prop}[Satz von H.A. Schwartz]
Sei $\Omega$ offen, $f\in C^1(\Omega,\R^m)$, und $x,y$ seien zwei beliebige
Koordinaten in $\Omega$. Existiert dann die zweite Abbleitung $f_{xy}$ und ist
sie dort stetig, so existiert auch $f_{yx}$ und es gilt $f_{xy} = f_{yx}$.
\end{prop}

\subsection{Sätze für Funktionen $f: V\hookrightarrow\R$}
\begin{prop}
Im Standardfall ist der Gradient einer Abbildung $f: \R^n\hookrightarrow\R$
der Spaltenvektor
\begin{align*}
\nabla f(x) = \begin{pmatrix} \partial_{x_1} f(x) \\ \vdots \\ \partial_{x_n}
f(x)
\end{pmatrix} = Df(x)^\top.
\end{align*}
\end{prop}
\begin{prop}[Kettenregel]
Ist $g: \R^n\hookrightarrow\R^m$ differenzierbar in $a$ und $f:
\R^n\hookrightarrow\R$ in $f(a)$, so gilt
\begin{align*}
\nabla (f\circ g)(a) = Dg(a)^\top\nabla f(g(a)).
\end{align*}
\end{prop}
\begin{prop}[Produktregel]
Seien $f,g: V\hookrightarrow W$ differenzierbar, $\lin{\cdot,\cdot}: W\to
\R$ eine Bilinearform und $\ph(x) = \lin{f(x),g(x)}$. Dann ist $\ph$
differenzierbar und es gilt
\begin{align*}
\D\ph(x) = \lin{\D f(x)\cdot, g(x)} + \lin{f(x), \D g(x)\cdot}. 
\end{align*}
\end{prop}
\begin{prop}[Mittelwertsatz]
Ist $f \in C^1(\Omega,\R)$ und $[u,v]\subset\Omega$, dann existiert ein
$\xi\in[u,v]$ sodass
\begin{align*}
f(v)-f(u) = Df(\xi)(v-u) = \lin{\nabla f(\xi),v-u}.
\end{align*}
\end{prop}
\begin{prop}
Ist $f: \R^n\hookrightarrow\R$ in $x$ stetig differenzierbar und $\nabla
f(x)\neq 0$, so bezeichnet $\nabla f(x)$ die Richtung des steilsten Anstiegs
und $-\nabla f(x)$ die Richtung des steilsten Abstiegs. Beide Richtungen sind
eindeutig.
\end{prop}
\begin{prop}
Sei $Q := [a,b]\times[c,d]$ und sei $g: Q\to\R$ auf $Q$ stetig und stetig nach
$y$ differenzierbar, dann ist auch
\begin{align*}
\ph: Q\to\R,\quad\ph(x,y) = \int_a^x g(s,y)\ds,
\end{align*}
nach $y$ differenzierbar und es gilt
\begin{align*}
\ph_y(x,y) = \int_a^x g_y(s,y)\ds.
\end{align*}
\end{prop}

