\section{Stetigkeit}

\begin{defn} Eine Abbildung $f: D\to W$ heißt \emph{stetig im Punkt} $a\in V$,
wenn es zu jedem $\ep > 0$ ein $\delta > 0$ gibt, sodass
\begin{align*}
&\norm{f(x)-f(a)}_W < \ep,\quad x\in B_\delta(a) \cap D.
\end{align*}
Mit Quantoren ausgedrückt
\begin{align*}
&\forall \ep > 0 \exists \delta > 0 \forall x\in B_\delta(a) \cap D :
\norm{f(x)-f(a)}_W < \ep.
\end{align*}
Ist $f$ in $a$ nicht stetig, heißt sie dort \emph{unstetig}.
\end{defn}
\begin{defn}
Eine Abbildung heißt \emph{stetig}, wenn sie in allen Punkten ihres
Definitionsbereichs stetig ist.
\end{defn}
\begin{defn} Eine Abbildung $f: D\to W$ heißt \emph{gleichmäßig stetig} auf $D$,
wenn
\begin{align*}
&\forall \ep > 0 \exists \delta > 0 \forall x_0,x \in D : \norm{x-x_0}_V <
\delta \Rightarrow \norm{f(x)-f(x_0)}_W < \ep.
\end{align*}
\end{defn}
\begin{defn} Eine Abbildung $f: D\to W$ heißt \emph{lipschitzstetig} auf $D$,
wenn es eine Konstante $L\ge 0$ gibt, sodass für $x,a\in D$ gilt
\begin{align*}
\norm{f(x)-f(a)}_W \le L \norm{x-a}_V.
\end{align*} 
Dabei ist die beste \emph{Lipschitz-Konstante}
\begin{align*}
L = \sup \limits_{\atop{x,y\in
D}{x\neq y}} \frac{\norm{f(x)- f(y)}_W}{\norm{x-y}_V}.
\end{align*}
\end{defn}
% \begin{defn}
% Eine Funktion $f: I \to \R$ heißt streng monoton fallend, falls
% \begin{align*}
% u < v \Rightarrow f(u) < f(v).
% \end{align*}
% \end{defn}
\begin{defn}
Den \emph{Raum der Funktionen} auf $D$ bezeichnet man mit \emph{$F(D)$}.
\end{defn}
\begin{defn}
Den \emph{Raum der beschränkten Funktionen} auf $D$ bezeichnet man mit
\emph{$B(D)$}.
\end{defn}
\begin{defn}
Den \emph{Raum der stetigen Funktionen} auf $D$ bezeichnet man mit
\emph{$C(D)$}.
\end{defn}
\begin{defn}
Den \emph{Raum der beschränkten stetigen Funktionen} auf $D$ bezeichnet man mit
\emph{$CB(D)$}.
\end{defn}
\begin{defn}
Eine Folge $(f_n)$ in $F(D)$ \emph{konvergiert punktweise} gegen eine Funktion
$f\in F(D)$, falls
\begin{align*}
\forall x \in D \forall \ep > 0 \exists N_0 \ge 0 : \norm{f_n(x)-f(x)} <
\ep.
\end{align*}
\end{defn}
\begin{defn}
Eine Folge $(f_n)$ in $F(D)$ \emph{konvergiert gleichmäßig} gegen eine Funktion
$f\in F(D)$, falls
\begin{align*}
\forall \ep > 0 \exists N_0 \ge 0 : \norm{f_n - f} < \ep.
\end{align*}
\end{defn}

\subsection{Beispiele stetiger Funktionen}

\begin{enumerate}
  \item Auf einem beliebigen normierten Raum sind die $\id$ und die konstanten
Funktionen lipschitz.
  \item Auf $\C$ sind $\Re z$ und $\Im z$ lipschitz.
  \item Jede Norm $\norm{\cdot}$ auf dem $\R^n$ ist lipschitz.
  \item Jedes reelle Polynom ist stetig.
  \item Die Wurzelfunktion ist stetig auf $[0,\infty)$ und für jedes $b>0$ auf
  $[0,b]$ gleichmäßig aber nicht lipschitzstetig.
\end{enumerate}

\subsection{Sätze für Funktionen in beliebigen normierten Räumen}

\begin{prop}[Folgenkriterium]
Eine Funktion $f: D\to F$ ist stetig in $a$ genau dann, wenn für jede Folge
$x_n \to a$ gilt $f(x_n) \to f(a)$.
\end{prop}
\begin{prop}[Folgenkriterium für Grenzwerte]
Ist $f: D\to F$ stetig, und $a\in D$ dann gilt 
\begin{align*}
\lim \limits_{x\to a} f(x) = w,
\end{align*}
genau dann wenn für jede Folge $x_n \to a, x_n\neq a$ gilt
\begin{align*}
\lim \limits_{n\to\infty} f(x_n) = w.
\end{align*}
\end{prop}
\begin{prop}
$C(D)$, ist eine Algebra.
\end{prop}
\begin{prop}
Ist $f$ auf $D$ stetig und $g$ auf einer Obermenge von $f(D)$, dann ist $g\circ
f$ ebenfalls stetig.
\end{prop}
\begin{prop}
$f: D\to F$ ist stetig in $a$ genau dann, wenn $f^{-1}(U_\ep(f(a)))$ für jedes
$\ep > 0$ eine $D$-relative Umgebung $U_\delta(a)$ enthält.
\end{prop}
% \begin{prop}
% $f: D\to F$ ist stetig auf $D$ genau dann, das Urbild $f^{-1}(A)$ jeder
% abgeschlossenen Menge $A$ abgeschlossen ist.
% \end{prop}
\begin{prop}
$f: D\to F$ ist stetig auf $D$ genau dann, das Urbild $f^{-1}(A)$ jeder
offenen Menge $A$ offen ist.
\end{prop}
\begin{prop}
Ist $K\subset E$ kompakt und $f: E\to F$ stetig, dann ist auch $f(K)$
kompakt.
\end{prop}
\begin{prop}
Ist $K\subset E$ kompakt und $f: E\to F$ stetig, dann ist $f\big|_K$
gleichmäßig stetig.
\end{prop}
\begin{prop}
Sind $M,N$ kompakt und $f: M\to N$ bijektiv und stetig, dann ist $f^{-1}$
ebenfalls stetig.
\end{prop}
\begin{prop}
Ist $K\subset E$ kompakt und $f: K\to \R$ stetig, dann nimmt $f$ sein Minimum
und sein Maximum an.
\end{prop}
\begin{prop}
Ist $a$ ein Häufungspunkt im Definitionsbereich der Funktion $f: D\to F$, und
existiert der Grenzwert $\lim \limits_{x\to a} f(x) = w$, so ist die Funktion
\begin{align*}
\tilde{f}: D\cup \{a\} \to F,\quad \tilde{f}(x) = \begin{cases}w, & \text{für }
x = a\\ f(x), & \text{sonst},
       \end{cases}
\end{align*}
stetig in $a$.
\end{prop}

\subsection{Sätze für Funktionen in Banachräumen}

\begin{prop}
Konvergiert eine Funktionenfolge $(f_n)$ in $F(D)$ gleichmäßig gegen $f$ und
sind alle $f_n$ stetig (integrierbar), dann ist auch $f$ stetig
(integrierbar).
\end{prop}
\begin{prop}
Konvergiert eine Folge $(f_n)$ differenzierbarer Funktionen in $F(D)$ punktweise
gegen $f$ und die Folge der Ableitungen $(f_n')$ gleichmäßig gegen $g$, dann ist
$f$ differenzierbar mit Ableitung $g$.
\end{prop}
\begin{prop}
$B(D)$ und $CB(D)$ sind mit der Supremumsnorm Banachräume.
\end{prop}
\begin{prop}
Ist $K$ eine kompakte Menge, dann ist $C(K)$ mit der Supremumsnorm
Banachraum.
\end{prop}
\begin{prop}
Ist $f: D\to F$ lipschitz, dann existiert genau eine stetige Forstetzung 
\begin{align*}
\Phi: D^- \to F,\quad \Phi|_D = f,
\end{align*}
von $f$. Sie ist lipschitz mit der selben $L$-Konstante wie $f$.
\end{prop}

\subsection{Sätze für reellwertige Funktionen}

\begin{prop}
Ist $K\subset E$ kompakt und $f: K\to \R$ stetig, dann ist $f$ gleichmäßig
stetig.
\end{prop}
\begin{prop}
Sind $f,g$ stetige Funktionen und $\lim \limits_{x\to a} f(a) = u, \lim
\limits_{x\to a} g(a) = v$, dann gilt
\begin{enumerate}
  \item $\lim \limits_{x\to a} (\lambda f + \mu g)(a) = \lambda u + \mu v$.
  \item $\lim \limits_{x\to a} (fg)(a) = uv$.
  \item $\lim \limits_{x\to a} (fg^{-1})(a) = uv^{-1}$, für $v\neq 0$.

  Gilt außerdem in einer punktierten Umgebung von $a$ $f(x) \le g(x)$, dann gilt
  \item $\lim \limits_{x\to a} f(a) \le \lim \limits_{x\to a} g(a)$.
\end{enumerate}
\end{prop}

\subsection{Sätze für Funktionen auf einem Intervall $f: I\to\R$}

\begin{prop}[Zwischenwertsatz von Bolzano]
Ist  $f: [a,b]\to \R$ stetig und $f(a)<f(b)$, dann gibt es zu jedem $w\in
\R: f(a) \le w \le f(b)$ ein $c\in[a,b]$, sodass $f(c) = w$.
\end{prop}
\begin{prop}
Ist $I$ ein Intervall und $f: I\to\R$ stetig, dann nimmt $f$ jeden
Wert zwischen $\inf f$ und $\sup f$ mindestens einmal an.
\end{prop}
\begin{prop}
Ist $I$ ein Intervall und $f: I\to\R$ stetig, dann ist auch $f(I)$ ein
Intervall.
\end{prop}
\begin{prop}[Satz über stetige Umkehrfunktionen]
Ist $I$ ein Intervall, $f: I\to\R$ stetig und streng monoton steigend, dann
gilt
\begin{enumerate}
  \item $f(I) = J$ ist ein Intervall.
  \item $f: I\to J$ ist bijektiv und besitzt eine Umkehrfunktion $f^{-1}: J \to
  I$.
  \item $f^{-1}$ ist stetig und streng monoton steigend auf $J$.
\end{enumerate}
\end{prop}
\begin{prop}
Für $n\ge 2$ besitzt die Funktion $t\mapsto t^n$ eine stetige streng monoton
steigende Umkehrfunktion, die $n$-te Wurzel.
\end{prop}
\begin{prop}
Ist $f: \R\to\R$ monoton, so existieren in jedem Punkt die links- und
rechtsseitigen Grenzwerte von $f$.
\end{prop}
% \begin{prop}
% Ist $f: [a,b]\to \R$ stetig, dann existieren $u,v\in[a,b]$, sodass
% $f(u)\le f(x)\le f(v)$.
% \end{prop}
% \begin{prop}
% Eine stetige Funktion auf einem abgeschlossenen Intervall ist beschränkt und
% bildet dieses wieder auf ein abgeschlossenes Intervall ab.
% \end{prop}